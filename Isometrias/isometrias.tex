%Encabezado estándar
\documentclass[10pt,a4paper]{article}
\usepackage[utf8]{inputenc}
\usepackage{amsmath}
\usepackage{amssymb}
\usepackage{amsthm}
\usepackage{hyperref}
\usepackage{graphicx}
\usepackage{subfigure} %paquete para poder añadir subfiguras a una figura
\usepackage{color}
\usepackage{float}
\usepackage[toc,page]{appendix} %paquete para hacer apéndices
\usepackage{cite} %paquete para que Latex contraiga las referencias [1-4] en lugar de [1][2][3][4]
\usepackage[nonumberlist]{glossaries} %[toc,style=altlistgroup,hyperfirst=false] 
%usar makeglossaries grafo para recompilar el archivo donde están los grafos y que así salga actualizado
\author{Alejandro García Montoro \\ Serafín Moral García \\ Gustavo Rivas Gervilla \\ Luis Suárez Lloréns}
\title{Isometrías líneales}
\date{}

%Configuración especial
\setlength{\parindent}{0cm}
\pretolerance=10000
\tolerance=10000

\begin{document}
\maketitle
\tableofcontents

\section{Introducción}

A lo largo de este trabajo veremos un tipo muy particular de aplicaciones lineales, las isometrías. Que como sabemos en esencia son funciones que conservan distancias en espacios vectoriales euclídeos. Haremos un repaso de las isometrías en $\mathbb{R}^2$ y en $\mathbb{R}^3$ viendo las matrices asociadas a cada una de ellas.\\

Además también resolveremos algunos problemas relacionados con las isometrías con el objetivo final de dar un repaso general a estas aplicaciones que nos será de utilidad a lo largo de la asignatura.\\

Seguiremos la estrutura presente en \cite{merino} para presentar estos conceptos ya que es un libro de uso muy extendido y de muy buena calidad.

\section{Definición y ejemplos}



\newpage
%Bibliografia
\nocite{*}
\bibliographystyle{unsrt}
\bibliography{biblio}

\end{document}