%Encabezado estándar
\documentclass[10pt,a4paper]{article}
\usepackage[utf8]{inputenc}
\usepackage{amsmath}
\usepackage{amssymb}
\usepackage{amsthm}
\usepackage{hyperref}
\usepackage{graphicx}
\usepackage{subfigure} %paquete para poder añadir subfiguras a una figura
\usepackage{color}
\usepackage{float}
\usepackage[toc,page]{appendix} %paquete para hacer apéndices
\usepackage{cite} %paquete para que Latex contraiga las referencias [1-4] en lugar de [1][2][3][4]
\usepackage[nonumberlist]{glossaries} %[toc,style=altlistgroup,hyperfirst=false] 
%usar makeglossaries grafo para recompilar el archivo donde están los grafos y que así salga actualizado
\author{Alejandro García Montoro \\ Serafín Moral García \\ Gustavo Rivas Gervilla \\ Luis Suárez Lloréns}
\title{Isometrías líneales}
\date{}

%Configuración especial
\setlength{\parindent}{0cm}
\pretolerance=10000
\tolerance=10000

%Configuración para la apariencia de definiciones y teoremas.
\newtheoremstyle{mystyle}% name
  {3pt}%Space above
  {3pt}%Space below
  {\normalfont}%Body font
  {0pt}%Indent amount
  {\itshape}% Theorem head font
  {}%Punctuation after theorem head
  {\newline}%Space after theorem head 2
  {}%Theorem head spec (can be left empty, meaning ‘normal’)

\theoremstyle{mystyle}
\newtheorem{defi}{\textcolor{red}{\textbf{Definición}}}
\newtheorem{teo}{\textcolor{blue}{\textbf{Teorema}}}


\begin{document}
\maketitle

\begin{figure}[H]
\centering
\includegraphics[width=70mm]{escudo.jpeg}
\end{figure}

\newpage

\tableofcontents

\newpage

\section{Introducción}

A lo largo de este trabajo veremos un tipo muy particular de aplicaciones lineales: las isometrías. Como sabemos, en esencia son funciones que conservan distancias en espacios vectoriales euclídeos. Haremos un repaso de las isometrías en $\mathbb{R}^3$ viendo las matrices asociadas a cada una de ellas.\\

Además, también resolveremos algunos problemas relacionados con las isometrías con el objetivo final de dar un repaso general a estas aplicaciones. Esto nos será de utilidad a lo largo de la asignatura.\\

Seguiremos la estructura presente en \cite{merino}, un libro de uso muy extendido y de muy buena calidad, para presentar estos conceptos.

\section{Definición y ejemplos}

\fbox{
\begin{minipage}{\textwidth}

\begin{defi}[Isometría]
\label{defiisometria}
Una aplicación lineal $f \, : \, V \rightarrow V'$ es \textbf{isometría} si, y sólo si, $\lVert v \rVert = \lVert f(v) \rVert$ para todo $v \in V$.
\end{defi}
\end{minipage}
}
\hfill \\

De esta definición podemos deducir que cualquier isometría es una aplicación inyectiva. Para ello sólo tenemos que ver que el núcleo de la aplicación es el conjunto formado únicamente por el cero: Si $f(v) = 0 \implies \lVert v \rVert= \lVert f(v) \rVert = 0 \implies v = 0$, c.q.d.

Por tanto, si $V$ y $V'$ tienen la misma dimensión (en particular si $f$ es un endomorfismo), toda isometría de $V$ a $V'$ es un isomorfismo.\\

A continuación vamos a ver una de las caracterizaciones más importante de las isometrías: como sabemos, toda aplicación líneal tiene asociada una matriz; pues bien, veamos qué propiedad tienen las matrices asociadas a isometrías:\\

\fbox{
\begin{minipage}{\textwidth}

\begin{teo}[Caracterización de las isometrías]
\label{teo1}
Si $V$ es un espacio vectorial euclídeo de dimensión finita y $B$ es una base ortonormal de V, entonces un endomorfismo f  de $V$ es una isometría si, y sólo si, su matriz asociada respecto de $B$ es ortogonal.
\end{teo}
\end{minipage}
}
\hfill \\

\textbf{DEMOSTRACIÓN:}\\

Si $f$ es isometría y $A$ su matriz asociada respecto de B, entonces se verifica:\\

\begin{center}
$X^tIY = <x,y> = <f(x),f(y)> = (AX)^tI(AY) = X^t(A^tIA)Y$
\end{center}

donde $X$ e $Y$ representan las coordenadas de los vectores de x e y respecto de la base B. Como esto se da para todo par de vectores, entonces tenemos que $A^tA = I$, con lo cual $A$ es ortogonal. El recíproco es evidente. $\blacksquare$

\section{Isometrías en $\mathbb{R}^3$}

A continuación vamos a estudiar los diferentes tipos de isometrías que hay en el espacio vectorial euclídeo $\mathbb{R}^3$.\\

\subsection{Giro de ángulo $\alpha$ respecto de una recta}

Vamos a considerar una base ortonormal de $\mathbb{R}^3$ compuesta por dos vectores $a$ y $b$ en el plano perpendicular al eje de giro y un vector $n$ perteneciente al eje de giro ($n = axb$). La matriz asociada al giro de ángulo $\alpha$ respecto a la recta de vector director $n$ sería:\\

\[
A=A_{\alpha}=
\begin{bmatrix}
\cos(\alpha) & -\sin(\alpha) & 0 \\
\sin(\alpha) &  \cos(\alpha)  & 0 \\
0 & 0 & 1 \\
\end{bmatrix}
\]


Algunas propiedades que observamos de la matriz anterior:\\ 

\begin{enumerate}
	\item 
	Las filas y columnas de $A$ forman una base ortonormal de $\mathbb{R}^3$, lo cual es consistente con que $A$ es una matriz ortogonal.
	
	\item
	
	El giro respecto de una recta en $\mathbb{R}^3$ es una isometría directa; esto es, con determinante -1 (el determinante no depende de la base escogida).
	
	\item
	
	La composición de giros respecto de una misma recta es también un giro. Además, si $\alpha$ y $\beta$ son ángulos, se cumplen las siguientes propiedades:
	
	\begin{enumerate}
		\item
		$A_{\alpha} A_{\beta} = A_{\alpha+\beta}$
		\item
		$A_{\alpha} A_{-\alpha} = I $
		\item
		$A_{\alpha}^{-1} = A_{-\alpha}$
	\end{enumerate}
	
	\item
	
	El eje de giro es un subespacio propio de valor propio asociado $\lambda = 1$, ya que los vectores del eje de giro se quedan invariantes.
	
	\item
	
	El plano perpendicular al eje de giro es un subespacio invariante, por lo que la imagen de
	un vector de ese plano permanece en ese plano.
	
	\item
	
	Para $\alpha = 0$ tenemos la matriz identidad.
	
\end{enumerate}

\subsection{Simetría respecto de una recta}

Es un caso particular del tipo de isometría anterior tomando como ángulo $\alpha = \pi$. La matriz en este caso sería  
\[
A=
\begin{bmatrix}
-1 & 0 & 0 \\
0 &  -1  & 0 \\
0 & 0 & 1
\end{bmatrix}
\]


Se verifica que 

\begin{enumerate}
	\item
	$ A_\pi A_\pi = I $
	\item
	$ A_\pi = A_{\pi}^{-1} $
		
\end{enumerate}

\newpage
%Bibliografia
\nocite{*}
\bibliographystyle{unsrt}
\bibliography{biblio}

\end{document}