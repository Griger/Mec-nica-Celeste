%Encabezado estándar
\documentclass[10pt,a4paper]{article}
\usepackage[utf8]{inputenc}
\usepackage{amsmath}
\usepackage{amssymb}
\usepackage{amsthm}
\usepackage{hyperref}
\usepackage{graphicx}
\usepackage{subfigure} %paquete para poder añadir subfiguras a una figura
\usepackage{color}
\usepackage{float}
\usepackage[toc,page]{appendix} %paquete para hacer apéndices
\usepackage{cite} %paquete para que Latex contraiga las referencias [1-4] en lugar de [1][2][3][4]
\usepackage[nonumberlist]{glossaries} %[toc,style=altlistgroup,hyperfirst=false] 
%usar makeglossaries grafo para recompilar el archivo donde están los grafos y que así salga actualizado
\author{Alejandro García Montoro \\ Serafín Moral García \\ Gustavo Rivas Gervilla \\ Luis Suárez Lloréns}
\title{Isometrías lineales}
\date{}

%Configuración especial
\setlength{\parindent}{0cm}
\pretolerance=10000
\tolerance=10000

%Configuración para la apariencia de definiciones y teoremas.
\newtheoremstyle{mystyle}% name
  {3pt}%Space above
  {3pt}%Space below
  {\normalfont}%Body font
  {0pt}%Indent amount
  {\itshape}% Theorem head font
  {}%Punctuation after theorem head
  {\newline}%Space after theorem head 2
  {}%Theorem head spec (can be left empty, meaning ‘normal’)

\theoremstyle{mystyle}
\newtheorem{defi}{\textcolor{red}{\textbf{Definición}}}
\newtheorem{teo}{\textcolor{blue}{\textbf{Teorema}}}
\newtheorem{prop}{\textcolor{green}{\textbf{Proposición}}}

\begin{document}
\maketitle

\begin{figure}[H]
\centering
\includegraphics[width=70mm]{escudo.jpeg}
\end{figure}

\newpage

\tableofcontents

\newpage

\section{Introducción}

A lo largo de este trabajo veremos un tipo muy particular de aplicaciones lineales: las isometrías. Como sabemos, en esencia no son más que funciones que conservan distancias en espacios vectoriales euclídeos. Haremos un repaso de las isometrías en $\mathbb{R}^3$, viendo las matrices asociadas a cada una de ellas.\\

Además, resolveremos algunos problemas relacionados con las isometrías. Nuestro objetivo final es dar un repaso general de estas aplicaciones que nos será de utilidad a lo largo de la asignatura.\\

Seguiremos la estructura presente en \cite{merino}, un libro de uso muy extendido y de muy buena calidad, para presentar estos conceptos.
\section{Definición y ejemplos}

\fbox{
\begin{minipage}{\textwidth}

\begin{defi}[Isometría]
\label{defiisometria}
Una aplicación lineal $f \, : \, V \to V'$ es \textbf{isometría} si, y sólo si, $\lVert v \rVert = \lVert f(v) \rVert$ para todo $v \in V$.
\end{defi}
\end{minipage}
}
\hfill \\

Es claro que de esta definición podemos deducir que cualquier isometría es una aplicación inyectiva. Para ello sólo tenemos que ver que el núcleo de la aplicación es el conjunto formado únicamente por el cero: si $f(v) = 0 \implies \lVert v \rVert= \lVert f(v) \rVert = 0 \implies v = 0$, c.q.d. Por tanto, si $V$ y $V'$ tienen la misma dimensión ---en particular si $f$ es un endomorfismo---, toda isometría de $V$ a $V'$ es un isomorfismo.\\

A continuación vamos a ver una de las caracterizaciones más importante de las isometrías. Como sabemos, toda aplicación lineal tiene asociada una matriz; veamos entonces qué propiedad tienen las matrices asociadas a isometrías:\\

\fbox{
\begin{minipage}{\textwidth}

\begin{teo}[Caracterización de las isometrías]
\label{teo1}
Si $V$ es un espacio vectorial euclídeo de dimensión finita y $B$ es una base ortonormal de V, entonces un endomorfismo $f$  de $V$ es una isometría si, y sólo si, su matriz asociada respecto de $B$ es ortogonal.
\end{teo}
\end{minipage}
}
\hfill \\

\textbf{DEMOSTRACIÓN:}\\

Si $f$ es isometría y $A$ su matriz asociada respecto de B, entonces se verifica:\\

\begin{center}
$X^tIY = <x,y> = <f(x),f(y)> = (AX)^tI(AY) = X^t(A^tIA)Y$
\end{center}

donde $X$ e $Y$ representan las coordenadas de los vectores de x e y respecto de la base B. Como esto se da para todo par de vectores, entonces tenemos que $A^tA = I$, con lo cual $A$ es ortogonal. El recíproco es evidente. $\blacksquare$\\

Dada una isometría en un espacio vectorial $V$, podemos considerar el conjunto de los vectores que se mantienen invariantes por la isometría; esto es:\\

\begin{center}
$V_f = \lbrace v \in V \vert f(v) = v \rbrace$
\end{center}

que, evidentemente, es un subespacio vectorial de $V$. Si $A$ es la matriz asociada a $f$, entonces un vector $X$ está en $V_f$ sii $(AX = X) \Longleftrightarrow (A-I)X = 0$, lo que da lugar a un sistema homogéneo con matriz de coeficientes $A-I$. Por lo tanto, $dim V_f = n - rango(A-I)$.\\

Del mismo modo podemos definir el conjunto $V_{-f} = \lbrace v \in V \vert f(v) = -v \rbrace$,  que mediante el mismo razonamiento que hemos hecho antes tenemos que da lugar al sistema homogéneo $(A+I)X = 0$.\\

Vamos a probar ahora que en $\mathbb{R}^3$ al menos uno de los dos subespacios es no nulo.\\

\fbox{
\begin{minipage}{\textwidth}

\begin{prop}[]
\label{prop1}
Para una matriz ortogonal real de orden impar A, se verifica que $det(A-I) = 0$ ó $det(A+I) = 0$.
\end{prop}
\end{minipage}
}
\hfill \\

\textbf{DEMOSTRACIÓN:}\\

En primer lugar observemos que si C es una matriz antisimétrica de orden impar, entonces det(C) = 0; en efecto, como $-C = C^t$, $det(C) = det(C^t) = (-1)^ndet(C)$ y n es impar, entonces det(C) = det(-C), ergo det(C) = 0.\\

Ahora: $(A-I)(A+I) = A^2 - I = A(A - A^t)$, pero como $A-A^t$ es antisimétrica, tenemos que $det((A-I)(A+I)) = 0$, de donde deducimos lo que queremos.$\blacksquare$\\

\section{Isometrías en $\mathbb{R}^3$}

A continuación vamos a estudiar los diferentes tipos de isometrías que hay en el espacio vectorial euclídeo $\mathbb{R}^3$.\\

\subsection{Giro de ángulo $\alpha$ respecto de una recta}

Vamos a considerar una base ortonormal de $\mathbb{R}^3$ compuesta por dos vectores $a$ y $b$ en el plano perpendicular al eje de giro y un vector $n$ perteneciente al eje de giro ($n = axb$). La matriz asociada al giro de ángulo $\alpha$ respecto a la recta de vector director $n$ sería:\\

\[
A=A_{\alpha}=
\begin{bmatrix}
\cos(\alpha) & -\sin(\alpha) & 0 \\
\sin(\alpha) &  \cos(\alpha)  & 0 \\
0 & 0 & 1 \\
\end{bmatrix}
\]


Algunas propiedades que observamos de la matriz anterior:\\ 

\begin{enumerate}
	\item 
	Las filas y columnas de $A$ forman una base ortonormal de $\mathbb{R}^3$, lo cual es consistente con que $A$ es una matriz ortogonal
	
	\item
	
	El giro respecto de una recta en $\mathbb{R}^3$ es una isometría directa, esto es, con determinante 1 (el determinante no depende de la base escogida).
	
	\item
	
	La composición de giros respecto de una misma recta es también un giro. Además, si $\alpha$ y $\beta$ son ángulos, se cumplen las propiedades:
	
	\begin{enumerate}
		\item
		$A_{\alpha} A_{\beta} = A_{\alpha+\beta}$
		\item
		$A_{\alpha} A_{-\alpha} = I $
		\item
		$A_{\alpha}^{-1} = A_{-\alpha}$
	\end{enumerate}
	
	\item
	
	El eje de giro es un subespacio propio de valor propio asociado $\lambda = 1$, dado que los vectores del eje de giro se quedan invariantes.
	
	\item
	
	El plano perpendicular al eje de giro es un subespacio invariante, por lo que la imagen de
	un vector de ese plano permanece en ese plano.
	
	\item
	
	Para $\alpha = 0$ tenemos la matriz identidad.
	
\end{enumerate}

\subsection{Simetría respecto de una recta}

Es un caso particular del tipo de isometría anterior tomando como ángulo $\alpha = \pi$. La matriz en este caso sería  
\[
A=
\begin{bmatrix}
-1 & 0 & 0 \\
0 &  -1  & 0 \\
0 & 0 & 1
\end{bmatrix}
\]


Se verifica que 

\begin{enumerate}
	\item
	$ A_\pi A_\pi = I $
	\item
	$ A_\pi = A_{\pi}^{-1} $
		
\end{enumerate}

\subsection{Simetría ortogonal respecto de un plano}

Tomando una base ortonormal de $\mathbb{R}^3$ con los primeros vectores a y b en el plano de simetría, y el tercer vector n en el eje ortogonal a él, la matriz asociada a esta base es:

\[
T=
\begin{bmatrix}
1 & 0 & 0 \\
0 &  1  & 0 \\
0 & 0 & -1
\end{bmatrix}
\]

Se tiene que:

\begin{enumerate}
\item $TT^{t} = I$ es decir $T^{-1} = T^{t} = T$. Tiene como valores propios el 1 doble y el -1 simple.
\item Es una isometría inversa, con determinante -1.
\item El plano de simetría es el subespacio propio asociado al autovalor $\lambda = 1$.
\item El eje ortogonal es el asociado al autovalor $\lambda = -1$
\end{enumerate}

\subsection{Clasificación}

Sea $f:\mathbb{R} \to \mathbb{R}^3$ una aplicación líneal conservando el producto escalar, $B=\lbrace e_1,e_2,e_3 \rbrace$ una base ortonormal y llamemos A a la matriz asociada a $f$ en la base $B$. Podemos distinguir entonces los siguientes casos:\\

Si det(A-I) = 0 $\implies$ existen vectores fijos no nulos. Clasificamos entonces según la dimensión del subespacio de vectores fijos:\\

\begin{enumerate}
\item Si $dimV_f = 1$ tenemos un \textbf{giro de eje la recta $V_f$ y ángulo $\alpha$}.
\item Si $dimV_f = 2$ tenemos una \textbf{simetría respecto del plano $V_f$}.
\item Si $dimV_f = 3$ tenemos la \textbf{identidad}.
\end{enumerate}

Ahora, si $det(A-I) \neq 0$ entonces por \ref{prop1} tenemos que $det(A+I) = 0$ y por lo tanto el espacio $V_{-f}$ es no nulo. Y entonces tenemos los siguientes casos:

\begin{enumerate}
\item Si $dimV_f = 1$ entonces tenemos una \textbf{composición de una simetría respecto del plano perpendicular a la recta $V_f$ con el giro de eje dicha recta y ángulo $\alpha$.}
\item Si $dimV_{-f} = 2$, entonces podemos tomar una base ortonormal $B' = \lbrace v_1,v_2,v_3 \rbrace$ en la que $v_1,v_2 \in V_{-f}$ y por tanto $f(v_3)$ es perpendicular a ambos, es decir, $f(v_3) = cv_3$ y, como $v_3 \notin V_{-f}$, entonces c = 1, con lo que estaríamos en el caso det(A-I) = 0, previamente analizado.
\item Si $dimV_{-f} = 3$ la aplicación es la \textbf{-I}.
\end{enumerate}

\section{Aplicaciones en mecánica celeste}

Describimos a continuación dos propiedades que nos permitirán estudiar con comodidad las soluciones de las ecuaciones diferenciales que la mecánica celeste nos plantea.

\subsection{Isometrías aplicadas a soluciones de ecuaciones diferenciales}

La siguiente proposición y demostración se encuentran en \cite{celeste}\\

\fbox{
\begin{minipage}{\textwidth}

\begin{prop}[]
\label{prop2}
Sea $x:I \to \mathbb{R}^3\ \{0\}$ un movimiento en un campo de fuerzas centrales
\[
\ddot{x} = f(\lVert x \rVert)\frac{x}{\lVert x \rVert},
\]
y sea $A \in \mathcal{O}(3)$ una matriz ortogonal. Entonces
\begin{align*}
Ax : &I \to \mathbb{R}^3- \{0\}\\
&t \mapsto Ax(t)
\end{align*}

es también una solución.
\end{prop}
\end{minipage}
}
\hfill \\

\textbf{DEMOSTRACIÓN:}\\

La demostración de esta propiedad es clara dada la definición de isometría.\\

Comprobemos que $Ax$ es solución estudiando si cumple la ecuación diferencial del campo de fuerzas centrales. Para ello, derivamos dos veces y manipulamos el resultado:

\[
\ddot{(Ax(t))} = A\ddot{x}(t)
\]

Por ser $x(t)$ solución, tenemos que

 \begin{align*}
 \ddot{x} = f(\lVert x \rVert)\frac{x}{\lVert x \rVert} \implies \\
 \ddot{(Ax(t))} = Af(\lVert x(t) \rVert)\frac{x(t)}{\lVert x(t) \rVert}
 \end{align*}
 
Ahora, teniendo en cuenta que $f(\lVert x(t) \rVert)$ es una constante y que la isometría cumple la propiedad $\lVert x \rVert = \lVert f(x) \rVert = \lVert Ax \rVert$, podemos escribir lo que sigue:

\[
\ddot{(Ax(t))} = f(\lVert x(t) \rVert)\frac{Ax(t)}{\lVert x(t) \rVert} = f(\lVert Ax(t) \rVert)\frac{Ax(t)}{\lVert Ax(t) \rVert}
\]\\

Concluimos entonces que $Ax$ es también solución. $\blacksquare$

\subsection{Isometría entre planos con intersección no vacía}

El siguiente ejercicio se encuentra planteado en \cite{celeste}\\

\fbox{
\begin{minipage}{\textwidth}

Demuestra que dados dos planos $\pi$ y $\pi '$ que pasan por el origen, siempre existe una matriz $ A \, \in \mathcal{O} (3)$ tal que $A \pi \, = \, \pi '$. ¿Cuántas?

\end{minipage}
}
\hfill\\

\textbf{SOLUCIÓN:}\\

Dejaremos aparte el caso trivial de que ambos planos, $\pi$ y $\pi '$, sean iguales. Siendo así, sabemos que por tener un punto en común, los planos se cortan en una recta. En nuestro caso, la recta además pasa por el origen, pues sabemos que este punto pertenece a la intersección de los planos.\\

Sea r la recta en la que se cortan $\pi$ y $\pi '$, que pasa por el origen. Es claro que una rotación en torno al eje r nos puede llevar $\pi$ en $\pi '$. Sean $\vec{n}$ y $\vec{n'}$ los vectores normales de $\pi$ y $\pi '$ respectivamente. Entonces, el ángulo de rotación será:\\

\[
\ \cos \alpha \, = \, \dfrac{<\vec{n},\vec{n'}>}{\|\vec{n}\|\|\vec{n'}\|}
\]\\

Por tanto, existe una isometría que nos lleva $\pi$ en $\pi '$: la rotación de ángulo $\alpha$ en torno a r.\\

En cuanto a la cantidad de isometrías que pueden hacer esto, primero tenemos que ver que la composición de isometrías es isometría. Pero esto es muy sencillo desde un punto de vista matricial, pues sabemos que las matrices ortogonales forman un grupo con las multiplicación.\\

Una vez hemos visto esto, la idea es aplicar la isometría indicada arriba, y después aplicar una segunda isometría que nos deje $\pi '$ en $\pi '$. Por ejemplo, las rotaciones de cualquier ángulo entorno al eje que tiene como dirección $\vec{n'}$, la normal del plano $\pi '$.\\

Por tanto, tenemos infinitas isometrías distintas que nos llevan $\pi$ en $\pi '$.\\

\subsection{Conclusiones}
¿Cuál es entonces la ventaja de las dos propiedades anteriores? ¿En qué nos pueden ayudar a la hora de estudiar mecánica celeste?\\

La respuesta es clara: al poder \emph{mover} las soluciones como \emph{queramos}, y teniendo en cuenta que podemos \emph{intercambiar} cualesquiera dos planos que pasen por el origen, podemos llevarnos los movimientos de cualquier partícula en un campo de fuerzas centrales al plano que más nos convenga en cada caso ---normalmente, al plano $Z=0$---, pudiendo trabajar así en las coordenadas que nos sean más cómodas o naturales.\\

Entonces, el problema de estudiar los movimientos de los cuerpos celestes en todo el Universo podemos \emph{reducirlo} a estudiarlo en un plano previamente definido. Las ventajas son claras.

\newpage
%Bibliografia
\nocite{*}
\bibliographystyle{unsrt}
\bibliography{biblio}

\end{document}